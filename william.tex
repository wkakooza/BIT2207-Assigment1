\documentclass[options]{article}
\begin{document}
\textbf{A SOFTWARE TOOL TO MONITOR AND CONTROL BAKERY INVENTORY ACTIVITIES}
\section{\textbf{Introduction}}
The introduction of information communication technology (ICT) in an inventory management system of business organizations has helped to create an efficient and effective mode of business activity management (Kotler and Armstrong, 2004). The failure to adopt ICT in Inventory management in many business organizations has led to problems such as high costs, time wastage, wrong calculations, this is the case for Ever Brown Bakery which is still based on the manual method of inventory information management.
\section{\textbf{Background of the Study}}
Ever Brown Bakery is a limited company specializing in production and supply of bread, pies and cakes to the general public. The company is currently using a manual text file in management of organizational inventory information. The manual method has problems such as alteration of confidential data, wrong calculations, time wastage in information processing, high costs of maintenance and poor customer service.
\section{\textbf{General Objective}}
 The general objective was to develop a software tool to monitor and control bakery inventory activities using a case study of ever brown bakery so as to create an efficient and effective mechanism of monitoring business ac
\section{\textbf{Specific Objectives}}
The project study was guided by the following objectives:
1)	To review existing systems so as to identify requirements for the system to monitor and control bakery inventory activities.
2)	To design a model for the system.
3)	To implement the system.
4)	To test and validate the system
\section{\textbf{ Functional Scope}}
The system monitors and control bakery inventory activities, allows users to view items, add items, and add stock. It also allows administrators to add users delete users and also view reports in the stock production department
\section{\textbf{Significance of the Study}}
The results and findings of this study will help in; 
i.	Help in decision support making.
ii.	Reduce on the costs of maintenance.
iii.	Accessing of information to only authorized parties.
iv.	Contributing to literate to future researchers in the related field.
\section{\textbf{Literature Review}}
\subsection{\textbf{ Introduction}}
This chapter reviews the already existing literature related to the project of study. It will comprise of the information managements systems, benefits and challenges of information systems of Inventory management systems and the Existing Systems.
\section{\textbf{Benefits of Information Systems}}
Improved operational efficiency and flexibility, all business owners would want these. The more efficient and flexibly an operational then this indicates the low cost to run it. This can be achieved due to cut the bureaucracy in the company after the implementation of good information systems.
Improved quantity and quality of information, Information is an important component of business today. Who controls the information would act more responsive to changes and trends in the future. Application of good information system will certainly generate reports compilation of data that is managed by qualified and comprehensive database. This can be achieved for each of the reporting process is executed automatically by computer machines.
\section{\textbf{ Inventory management systems}}
An inventory system is a system that encompasses all aspects of managing a company's inventories; purchasing, shipping, receiving, tracking, warehousing and storage, turnover, and reordering. The activities associated with each of these areas may not be strictly contained within separate subsystems, but these functions must be performed in sequence in order to have a well-run inventory system (Colin, 2000). Computerized inventory management systems make it possible to integrate the various functional subsystems that are a part of the inventory management into a single cohesive system (Udo, 2003).
\section{\textbf{Conclusion}}
In summary therefore the team looked at information management systems, their benefits in business activities and the already existing systems (Murthy, 2006). This helped us to review the existing literature in accordance to Inventory management systems in order to implement a software tool to monitor and control bakery inventory activities.  
\section{\textbf{METHODOLOGY}}
\subsection{\textbf{3.0 Introduction}}
This chapter defines the detailed description of the step by step methods of how the proposed system objectives were achieved. It includes procedures and techniques that were used in data collection, design, implementation and finally testing and validation.
\section{\textbf{ Research Design}}
According to Garwood and Jupp (2011) a research design refers to a systematic plan drawn by the person carrying out research during the research study. Qualitative research design was employed to evaluate the use of a software tool to monitor and control bakery inventory activities. Qualitative research is that which investigates aspects of social life which are not amenable to quantitative measurement. The team adopted a qualitative research design to get experiences, viewpoints and suggestions towards reference services provided by Ever Brown Bakery. This research design is preferred because it was to help the team to study the selected issues in depth (Maggie and Jupp, 2011).
\section{\textbf{Population of the Study}}
A population is any set of persons or objects that possess at least one common characteristic and from the research can obtain information (Tripathi, 2005). The population of the study comprised of six staff members of Ever Brown Bakery in all the sections and forty users
\section{\textbf{Data Collection Methods}}
In gathering data relevant to the study, I used various methods of data collection and these include interview method, questionnaire method and literature search method in order to get information from the reliable sources.
\section{\textbf{Questionnaire}}
According to Mbaga and Kakinda (2000) a questionnaire is a set of related questions designed to collect information from a respondent. The team used both the open and close ended questionnaire so as to enable me to collect large amounts of data and also provide opportunity for respondents to give frank answers, twenty questionnaires were used and the team received sixteen answered questionnaires
\section{\textbf{Interview}}
Interview is a method of data collection that involves a face to face conversation between the interviewer and the respondent conducted for the purpose of obtaining information. (Mbaga and Kakinda, 2000) interviews can be either on telephone, personal as in face to face or even group interviews. This method was used on the Bakery staff where the team interacted with Fifteen staff members and requested for information to be used in the report. 
\section{\textbf{3.5 System Design}}
During system design, various tools were used to show how the system operates. The following system design methods were used;
\section{\textbf{The Context diagram:}}
 This was used to illustrate how the system interacts with external entities
\section{\textbf{Entity Relationship Diagram}}
In order to design a software tool to monitor and control bakery inventory activities, an entity relationship diagram was used to define different entities and their relationships in the new system developed. 
\section{\textbf{ Implementation}}
The system will be developed using PHP to link interfaces to the database to retrieve and store data, phpMyadmin, HTML 5 and Java script. 
\section{\textbf{Testing and validation}}
This is the process of executing application programs with the intent of finding errors. The faults were corrected and the process will be repeated until the system will proved to be working according to user’s specification and performance requirements. During the validation the team will check the system to ensure that it is working without malfunctions.

\section{\textbf{References}}
www.google.com
reportwrtting.com











                                                               






\end{document}